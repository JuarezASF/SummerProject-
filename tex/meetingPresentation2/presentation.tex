\documentclass{beamer}
\usepackage[T1]{fontenc}
\usepackage[utf8]{inputenc}
\usepackage{lmodern}
\usepackage{graphicx}			%para imagens
\usepackage{epstopdf} 			%resolve problemas eps-pdf
\usepackage{fancyhdr}			% para o cabeçalho bonito
\usepackage{caption}				%para legendas
\usepackage{placeins} 			%controlar o lugar dos floats
\usepackage{color}
\usepackage{url}
\renewcommand{\UrlFont}{\tiny}
\usepackage{relsize}
\usepackage{hyperref}

\usetheme{Dresden}
\usecolortheme{orchid}
\setbeamertemplate{navigation symbols}{}

\newcommand{\HRule}{\rule{\linewidth}{0.5mm}}

\title{Summer Research Project \\  Video Based Mouse Seizure Detection}
\author{Juarez Sampaio}
\titlegraphic{\leavevmode\smash{\raisebox{5.5cm}{\includegraphics[width=0.2\textwidth]{logo}}}}
\institute{Rice University}
\date{\today}


\begin{document}
\begin{frame}
        \titlepage
\end{frame}

\begin{frame}[fragile]
  \frametitle{Content}
  \begin{itemize}
     \item A few concepts in image processing and computer vision:
       \begin{itemize}
           \item Convolution
           \item Image pyramid
       \end{itemize}
     \item Using classification to tracking
     \item line wise machine learning approach to tracking
     \item Computing optical flow
  \end{itemize}
\end{frame}


\section{Convolution}
\subsection{Definition}
\begin{frame}
  \frametitle{\secname : \subsecname}
  \begin{itemize}
     \item Convolution is technique broadly use in image processing
     \item It consists of replacing the intensity value of every pixel by the result of an operation that involves a
       window around that pixel
      \item The operation is a sort of dot product between a kernel and the window. 
      \item Corresponding elements of the kernel and then window are multiplied and the added.
      \item In other words:
        \begin{equation}
          I(i,j) = dot(A, I[i-k:i+k, j-l:j+l])), \forall i,j 
        \end{equation}
  \end{itemize}
\end{frame}

\subsection{Example 1}
\begin{frame}
  \frametitle{\secname : \subsecname}
  \begin{figure}
    \centering
    \includegraphics[scale=0.4]{./convExample.png}
    \caption{Example of convolution to blur the image}
    \label{fig:pyr}
  \end{figure}
\end{frame}

\subsection{Example 2}
\begin{frame}
  \frametitle{\secname : \subsecname}
  \begin{figure}
    \centering
    \includegraphics[scale=0.6]{./convExample2.png}
    \caption{Example of convolution to compute derivatives. The upper filter computes x derivative while the bottom one
    computer y derivative. Both can be combined to find the gradient.}
    \label{fig:pyr}
  \end{figure}
\end{frame}
\section{Image Pyramid}

\subsection{Introduction}
\begin{frame}
  \frametitle{\secname : \subsecname}
  \begin{itemize}
      \item In many image processing tasks we want to work with different sizes of the same image. 
      \item The concept is as follows: the base of the pyramid contains the highest resolution image available and as
        we go up the pyramid the resolution decreases.
      \item The literature actually sees this process as an inverted pyramid. Thus, the high
        resolution image stands at the top and to decrease resolution we actually go down the pyramid.
    \end{itemize}
\end{frame}

\begin{frame}
  \frametitle{\secname : \subsecname}
  \begin{figure}
    \centering
    \includegraphics[scale=0.6, angle=180]{./Pyramids_Tutorial_Pyramid_Theory.png}
    \caption{Inverted pyramid. \footnote{source: open cv documentation
    \url{http://docs.opencv.org/doc/tutorials/imgproc/pyramids/pyramids.html}}}
    \label{fig:pyr}
  \end{figure}
\end{frame}

\subsection{Down sampling}
\begin{frame}
  \frametitle{\secname : \subsecname}
  Down sampling works as follows
  \begin{itemize}
        \item first we blur the image by convolving the image with a low pass kernel
        \item then we remove n-1 out of n rows and columns.
        \item The size of the blurring window depends on the degree n. Larger n implies larger blurring operation. 
        \item As in signal processing, the low pass operation is required to avoid alias.
  \end{itemize}
\end{frame}

\begin{frame}
  \frametitle{\secname : \subsecname}
  \begin{figure}
    \centering
    \includegraphics[scale=0.6, rotate=180]{./pyrDown_example.png}
    \caption{Down the pyramid}
    \label{fig:pyr_down}
  \end{figure}
\end{frame}

\subsection{Up sampling}
\begin{frame}
  \frametitle{\secname : \subsecname}
  Up sampling works as follows
  \begin{itemize}
        \item we add n-1 zeros between every element of the image
        \item we convolve the resulting image with the same low pass kernel
        \item This results into a interpolation of the original values
        \item The previous description allowed us to down sample by integer ratios D and up sample by U. Combining these two operations
          we can obtain any rational U/D. 
        \item Keep in mind that it is important to up sample first! Down sampling always implies loss of data!
  \end{itemize}
\end{frame}

\begin{frame}
  \frametitle{\secname : \subsecname}
  \begin{figure}
    \centering
    \includegraphics[scale=0.6, rotate=180]{./pyrUp_example.png}
    \caption{Up the pyramid}
    \label{fig:pyr_up}
  \end{figure}
\end{frame}


\subsection{Demo}
\begin{frame}
  \frametitle{\secname : \subsecname}
  \href{run:runPyramidDemo.sh}{Run pyramid demo}
\end{frame}

\section{Classification for Tracking}
\subsection{Generalizing the convolution}
\begin{frame}
  \frametitle{\secname : \subsecname}
  \begin{itemize}
      \item The convolution replaces the value of a pixel by the result of a dot product between a kernel and a window
        around the pixel. 
       \item What if instead of this dot product, the value of the pixel were replaced by the result of a function
          applied to a window around the pixel?
       \item This is precisely the idea I applied last week.
       \item The function used is a classifier of some sort. 
  \end{itemize}
\end{frame}

\subsection{Problem}
\begin{frame}
  \frametitle{\secname : \subsecname}
  \begin{itemize}
       \item The problem with this was that the libraries we're working with are not familiarized with this extension of
         the convolution concept and provide no easy way of realizing this classification.
       \item The convolution had to be hand implemented and the process became slow.
       \item Also, given the dimensionality of the problem, the accuracy rate of 94\% was not enough to allow this
         approach to reliably track the mouse
  \end{itemize}
\end{frame}

\section{Line wise machine learning approach to tracking}
\subsection{idea}
\begin{frame}
  \frametitle{\secname : \subsecname}
  \begin{itemize}
      \item The previous approach tried to use machine learning to answer the question 'is there a mouse in this
        square'?
      \item Now, we'll use machine learning to answer 'is there a mouse in this line'?
      \item Another classifier, trained in the same manner, will answer 'is there a mouse in this column'?
      \item Both classifiers will be solving a way smaller version of the problem. The row classifier will learn a
        function on space of dimensionality 320 and the column one on a space with d=240.
      \item As before, we can use image pyramids to reduce those dimensions to half and quarter of the full resolution!
    \end{itemize}
\end{frame}

\subsection{Details}
\begin{frame}
  \frametitle{\secname : \subsecname}
  \begin{itemize}
      \item The function to be learn has to detect something of the sort 'is there a sequence of consecutive pixels of
        similar intensity?'
      \item The value of a pixel alone has no meaning, it is its connections that matter. 
      \item Thus, there is a hidden feature to be learned here. This requires a classifier that is able to learn those
        hidden states.
      \item My bet is that neural networks are the proper tool to this task.
      \item If this approach works, the hidden nodes will learn the hidden properties and the output layer will combine
        them to produce the classification.
    \end{itemize}
\end{frame}

\begin{frame}
  \frametitle{Collecting samples}
  \href{run:runCollect.sh}{Run collect demo}
\end{frame}

\section{Optical Flow experiments}
\begin{frame}
  \frametitle{\secname}
  \begin{itemize}
      \item Optical flow is computed on the tracking window produced my the current best tracker
      \item Because we've reduced the problem to a 30x30 window, we can run dense optical flow.
      \item Currently I am only plotting the highest value of magnitude seen in the flow.
      \item The idea is that we can detect seizures by looking at the data
      \item Maybe we could divide this data into time slices of a given length and then compute its frequency
        components.
    \end{itemize}
  \href{run:runCollect.sh}{Run collect demo}
\end{frame}
\end{document}
