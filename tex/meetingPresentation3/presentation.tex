\documentclass{beamer}
\usepackage[T1]{fontenc}
\usepackage[utf8]{inputenc}
\usepackage{lmodern}
\usepackage{graphicx}			%para imagens
\usepackage{epstopdf} 			%resolve problemas eps-pdf
\usepackage{fancyhdr}			% para o cabeçalho bonito
\usepackage{caption}				%para legendas
\usepackage{placeins} 			%controlar o lugar dos floats
\usepackage{color}
\usepackage{url}
\renewcommand{\UrlFont}{\tiny}
\usepackage{relsize}
\usepackage{hyperref}

\usetheme{Dresden}
\usecolortheme{orchid}
\setbeamertemplate{navigation symbols}{}

\newcommand{\HRule}{\rule{\linewidth}{0.5mm}}

\title{Summer Research Project \\  Video Based Mouse Seizure Detection}
\author{Juarez Sampaio}
\titlegraphic{\leavevmode\smash{\raisebox{5.5cm}{\includegraphics[width=0.2\textwidth]{logo}}}}
\institute{Rice University}
\date{\today}


\begin{document}
\begin{frame}
        \titlepage
\end{frame}

\begin{frame}[fragile]
  \frametitle{Content}
  \begin{itemize}
     \item A few concepts in image processing and computer vision:
       \begin{itemize}
           \item Dot Product
           \item Discrete Fourrier Transform
       \end{itemize}
  \end{itemize}
\end{frame}


\section{Background Concepts}
\subsection{Inner Product}

\begin{frame}
  \frametitle{\secname : \subsecname}
  \begin{block}{Definition}
    The inner product is an operation obeying a set of properties that takes as input two vectors from a given space and produce a real number.
  \end{block}

  \begin{columns}[c]
    \begin{column}[l]{.5\textwidth}
      \begin{alertblock}{Application}
        An important application is to compute similarity between vectors.
        \begin{equation}
          A_b = \frac{||A\cdot B||}{||B||}
        \end{equation}
      \end{alertblock}
    \end{column}
    \begin{column}[l]{.5\textwidth}
      \begin{figure}
        \centering
        \includegraphics[width = 0.6\textwidth, keepaspectratio]{./dotProduct.png}
        \caption{Dot product can be used to compute the component of A in the direction of B.}
      \end{figure}
    \end{column}
   \end{columns}

\end{frame}

\begin{frame}
  \frametitle{\secname : \subsecname}
  A common space is the $\mathbb{R}^d$ and its usual dot product, but more interesting spaces also have inner product and
  different inner products can be defined for the same space, each one measuring a different kind of similarity. 
  \begin{itemize}
    \item space: $\mathbb{R}^d$
      \begin{equation}
        \vec{v} \cdot \vec{w} = \sum_{i=1}^{d} v_i w_i
      \end{equation}

      \item space: continuous real functions on interval (a,b)
      \begin{equation}
        <u,v> = \int_a^b u \cdot v
      \end{equation}

  \end{itemize}

\end{frame}

\begin{frame}
  \frametitle{\secname : \subsecname}
  \begin{block}{Application}
    Inner products are used to write a vector as a linear combination of a set of vectors that expand the space
  \end{block}
  \begin{exampleblock}{An example in $\mathbb{R}^2$}
    The vector v = (1,2) can be written as:
    \begin{equation}
      (1,2) = 1 . (1,0) + 2 . (0,1)
    \end{equation}
    Where (1,0) and (0,1) are vectors that expand $\mathbb{R}^2$. Note that the scalars can be found as:
    \begin{equation}
      (1,2) = [(1,2)\cdot(1,0)] . (1,0) + [(1,2)\cdot(0,1)] . (0,1)
    \end{equation}
  \end{exampleblock}
\end{frame}

\begin{frame}
  \frametitle{\secname : \subsecname}
  \begin{itemize}
    \item In general, let the inner product be written as $<\vec{a},\vec{b}>$ and let the set $\{\vec{v}_1, \ldots
      \vec{v}_d\}$ be the basis of a space of dimensionality d. A vector $\vec{x}$ can be written as:
      \begin{equation}
        \vec{x} = \sum_{i=1}^d <\vec{x},\vec{v}_i> \vec{v}_i
      \end{equation}
    \item call the result $<\vec{x},\vec{v}_i> = a_i$. We can represent x by the set of $a_i$'s. That is, we can
      reconstruct $\vec{x}$ from its components $a_i$'s
  \end{itemize}
\end{frame}


\subsection{Discrete Fourrier Transform}

\begin{frame}
  \frametitle{\secname : \subsecname}
  \begin{block}{Motivation}
    We want a technique that allows us to study properties of sequences. We're given a sequence and we'll perform a set
    of operations that will give us coefficients, each informing us some property of the original sequence
  \end{block}
  \begin{itemize}
    \item The sequence is an ordered and numbered set of values such as $\{ \ldots, a_{-1}, a_0, a_1,\ldots\}$
    \item We're going to write any sequence as a linear combination of some other sequences. 
    \item The basis of this space of sequences is a set of periodic complex exponentials $e^{jwn}$. 
    \item In order to write a sequence of length M we need at least M complex exponentials. They are
      $e^{j \frac{2\pi}{M}kn}, k = 0,1,\ldots,M-1$
  \end{itemize}
\end{frame}
\begin{frame}
  \frametitle{\secname : \subsecname}
  \begin{figure}
    \includegraphics[width=0.8\linewidth]{./sequencesM=12.png}
    \caption{A few sequences of the form $y_n = exp(j\frac{2\pi}{M}kn)$ for M = 12}
  \end{figure}
\end{frame}
\begin{frame}
  \frametitle{\secname : \subsecname}
  \begin{figure}
    \includegraphics[width=0.8\linewidth]{./fftSample.png}
    \caption{Example of 501 DFT of length 10 sequence}
  \end{figure}
\end{frame}

\begin{frame}
  \frametitle{\secname : \subsecname}
  \begin{exampleblock}{Applications}
    \begin{itemize}
        \item When transmitting audio, we can compute the DFT components and then transmit it. The receiver then
          reconstruct the original signal by adding the right amount of components. More quality implies more DFT
          components. 
        \item To every operation applied to the original signal, there is an equivalent operation applied to the
          signal's DFT. In some cases, it is easier and faster to convert the signal to its DFT, apply the operation on
          the frequency domain and then convert it back to time domain.
        \item Large convolutions are implemented this way by open cv.
    \end{itemize}
  \end{exampleblock}
\end{frame}

\begin{frame}
  \frametitle{\secname : \subsecname}
  \href{run:runFFT_Demo.sh}{Run FFT demo}
\end{frame}

\section{Week Results}
\subsection{Line Wise Detector}
\begin{frame}
  \frametitle{\secname : \subsecname}
  \begin{itemize}
      \item the neural network constructed was not able to learn the data. The accuracy I got on test data of around
        60\%
      \item The good news is that accuracy on training data was also very low. The good thing about it is that this
        suggests there is a bug somewhere, because neural networks should be able to, at least, overtfit the data. 
  \end{itemize}
\end{frame}

\subsection{Changing the reference axis to compute optical flow}
\begin{frame}
  \frametitle{\secname : \subsecname}
  \begin{itemize}
      \item Instead of computing the flow in the box, the flow is now computed in the points inside the box but in the
        global frame of the image.
      \item This reduced a lot the noise, but every now and then there is still some optical flow where nothing should
        be seen.
      \item Blurring the image only made it worse.
      \item \href{run:runFlow.sh}{Run demo}
  \end{itemize}
\end{frame}


\subsection{Working on the labels}
\begin{frame}
  \frametitle{\secname : \subsecname}
  \begin{itemize}
      \item At some point we'll have to look at the data we have
      \item I'm writing a program that takes as input a text file containing times of seizure and produces video
        containing it.
      \item The problem with this is that the data we have is split into several videos and some search is required.
  \end{itemize}
\end{frame}

\end{document}
