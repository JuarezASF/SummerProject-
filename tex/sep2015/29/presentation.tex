\documentclass{beamer}
\usepackage[T1]{fontenc}
\usepackage[utf8]{inputenc}
\usepackage{lmodern}
\usepackage{graphicx}			%para imagens
\usepackage{epstopdf} 			%resolve problemas eps-pdf
\usepackage{fancyhdr}			% para o cabeçalho bonito
\usepackage{caption}				%para legendas
\usepackage{placeins} 			%controlar o lugar dos floats
\usepackage{color}
\usepackage{url}
\renewcommand{\UrlFont}{\tiny}
\usepackage{relsize}
\usepackage{hyperref}

\usetheme{Dresden}
\usecolortheme{orchid}
\setbeamertemplate{navigation symbols}{}

\newcommand{\HRule}{\rule{\linewidth}{0.5mm}}

\title{Video Based Mice Seizure Detection}
\author{Juarez Sampaio}
\titlegraphic{\leavevmode\smash{\raisebox{5.5cm}{\includegraphics[width=0.6\textwidth]{./unbLogo.jpg}}}}
\institute{Universidade de Brasília}
\date{\today}


\begin{document}
\begin{frame}
        \titlepage
\end{frame}

\begin{frame}[fragile]
  \frametitle{Content}
  \begin{itemize}
     \item Ideas to discuss
       \begin{itemize}
           \item Dense optical flow is actually good
           \item What are the shadows in the Eulerian magnified video?
           \item Flow Difference
       \end{itemize}
  \end{itemize}
\end{frame}


\begin{frame}
  \frametitle{Dense optical flow is actually good}
  \only<1-4>{This came out from a discussion with Professor Fábio}
  \begin{itemize}

      \only<1-4>{
      \uncover<1-4>{\item Suppose we define a small region of the image to compute a sparse flow}
      \uncover<2-4>{\item The sparse flow computation will solve a minimization problem}
      \uncover<3-4>{\item Even though we are computing flow for a small region, we do send the entire
      image to the algorithm}
      \only<4-4>{\item Problem: a point inside the area of interest(small region) may be mapped to a point far a way
      from its previous known position.}
      }
      \uncover<5->{\item I'm looking for 20 points in a 10 km by 10 km square. At time t, point A is at position (0,0).
        At time t+$\Delta t$ I see a point that looks like A, but at position (10,10)km. Because I'm only
        looking for 20 points and I'm a minimization algorithm, I'll say A moved from (0,0) to (10,10) in
        $\Delta t$.}
      \uncover<6->{\item Suppose now that I'm looking for 1 million points inside that same square. At time $t +\Delta t$ it is very
        unlikely that anyone moved from (0,0) to (10,10). Thus, I'll discard this possibility and say that point A must
        be near its previous known position. Even though there's a point at (10,10) that looks exactly like A at the
        previous time, I'll won't consider it to be A, because it is too far away.}
  \end{itemize}
\end{frame}


\begin{frame}
  \frametitle{Dense optical flow is actually good}
  \begin{block}{The point is}
       We'll have less noise in optical flow computation if we use a dense grid.
  \end{block}
\end{frame}

\begin{frame}
  \frametitle{What are the shadows in the magnified video?}
  \begin{itemize}
      \only<1-2>{
      \uncover<1-2>{\item For a fixed pixel $(i,j)$ in the image, plot its intensity value as a discrete function
      $I_{i,j}(k)$ in the interval of interest of length T. You'll have $N = \frac{T}{\Delta T}$ points to plot,
      $\Delta T$ being the sampling time}
      \uncover<2>{\item This function can be decomposed into a sum of N sinusoidals inside the interval of interest
      \begin{equation}
        I_{i,j}(k) = \sum_{L=0}^{N-1} a_n \cos(\frac{2\pi}{N}L k) + b_n \sin(\frac{2\pi}{N}L k)
      \end{equation}
      }
      }
      \uncover<3->{\item Or, in complex exponential form
      \begin{equation}
        I_{(i,j)}(k) = \sum_{L=0}^{N-1} d_n e^{j\frac{2\pi}{N}L k}
      \end{equation}
      }
      \uncover<4->{\item The $d_n$ coefficients can be computed with the discrete Fourrier Transform.}
      \uncover<5>{\item Part of Eulerian magnification consists in applying a frequency filter the change de $d_n$
      coefficients as a function of n. The idea is to increase those coefficients so high frequency components are more
      obviously noted.}
  \end{itemize}
\end{frame}

\begin{frame}
  \frametitle{What are the shadows in the magnified video?}
  \begin{itemize}
      \only<1-3>{
      \uncover<1-3>{\item Remember however that we're dealing with pixel values. They are limited to a fixed range of
      values.}
      \uncover<2-3>{\item In openCV, for example, any value of pixel higher than 255 is set to 255. That is, 255 is the
      value for the brightest white, and nothing can be set brighter than the brightest white.}
      \uncover<3>{\item Our new function, after the magnification, is thus:
      \begin{equation}
        E_{i,j}(k) = min(\sum_{L=0}^{N-1} G(n)d_n e^{j\frac{2\pi}{N}L k}, 255)
      \end{equation}
      }
      }
      \only<4-5>{
      \uncover<4-5>{\item On the other hand, we cannot get darker than the darkest black(pixel value = 0)}
      \uncover<5>{ \item Thus, we correct again:
      \begin{equation}
        E_{i,j}(k) = max(0,min(\sum_{L=0}^{N-1} G(n)d_n e^{j\frac{2\pi}{N}L k}, 255))
      \end{equation}
      }
      }
      \uncover<6->{ \item I propose that the shadows we're observing is a result of the saturations of the system. The gain
      being applied is so high that any slight increase in intensity value produces a bright white and any slight
      decrease produces a dark black.}
      \uncover<7->{ \item When a mouse(white object) moves from A to B, point A became darker because it no longer hold a
      white pixel and point B hold now something brighter than previously it did. Due to saturation, A becomes black and
      B becomes white}
      \uncover<8>{ \item Ideally, only a pixel belonging to the fastest movement in the video should be mapped to 255.}
  \end{itemize}
\end{frame}

\begin{frame}
  \frametitle{Solving shadows}
  \begin{block}{Solution}
        Reduce frequency gain in order to avoid saturation.
  \end{block}
\end{frame}

\begin{frame}
  \frametitle{Flow Difference}
  This is Fábio's approach to study the seizure
  \begin{itemize}
      \only<1>{
      \item Our previous idea was that the flow field of a seizure event would have a high frequency
      component.
      \item The new approach is based on that same assumption, we simply use a different way of looking at it
      \item First we compute the vectorial difference between two consecutive flows. This correspond, in some sense I'm
        sure, to the spatial second derivative of the image.
      \item Remember that the laplacian operations corresponds to a high pass filter. We're here looking to the effect
        of the high frequency components
      \item We now construct an image with magnitude of the flow difference at each point
      \item We take the 2 dimensional Fourrier Transform of this image.
        }
      \only<2>{\item Vidal says that during seizures, it will be easy do detect high frequency by looking at points of the
      magnitude spectrum distant from the origin.}
  \end{itemize}
\end{frame}

\begin{frame}
  \frametitle{Flow Difference}
  My comments on the approach:
  \begin{itemize}
      \only<1-3>{\uncover<1-3>{
            \item Disadvantage: In order to compute the FFT of the magnitude image, it is interesting if we compute the flow
      on a dense grid. On my current demo, I compute flow for every point.
      }\uncover<2-3>{
      \item The problem of a sparse grid is that you'll have a flow difference magnitude image with spikes: it can only
        be different than zero in points where you actually computed the flow.
        }\uncover<3-3>{
      \item Experimentally(and it reminded me of results in one dimensional FFT, and I'm sure this is can be shown for
        two dimensional) by running FFT on a 2 dimensional image with spikes here and there, you get a FFT that has
        spikes over the entire spectrum. I feel it is also periodic. 
        }}

      \uncover<4->{
            \item I suggest that we can still run it on a sparse grid. When running the FFT though, we downsample the
              flow difference magnitude image so it only contains points where we computer the flow. 
      }\uncover<5->{
      \item I have implemented demos of the proposed methodoly and, so far, tested it on a 5 sec video. As you can
        imagine, computing a dense flow, taking its difference, its magnitude and then its DFT is computationally heavy
        and it becomes cumbersome to run experiments.
        }\uncover<6->{
      \item I have not showed Vidal the results so far. More experiments are required before we can drawn any
        conclusion. 
        }
  \end{itemize}
\end{frame}


\begin{frame}
  \frametitle{Conclusion}
  \begin{block}{}
    Dr. Vidal is proposed a methodology based on the 2 dimensional fourrier transform of the flow difference. The method
    is terribly heavy and needs more results so we evaluate if it is worth the trouble.
  \end{block}
\end{frame}
\end{document}
