\documentclass{beamer}
\usepackage[T1]{fontenc}
\usepackage[utf8]{inputenc}
\usepackage{lmodern}
\usepackage{graphicx}			%para imagens
\usepackage{epstopdf} 			%resolve problemas eps-pdf
\usepackage{fancyhdr}			% para o cabeçalho bonito
\usepackage{caption}				%para legendas
\usepackage{placeins} 			%controlar o lugar dos floats
\usepackage{hyperref}
\usepackage{color}

\usetheme{Dresden}
\usecolortheme{orchid}
\setbeamertemplate{navigation symbols}{}

\newcommand{\HRule}{\rule{\linewidth}{0.5mm}}

\title{Summer Research Project \\  Video Based Mouse Seizure Detection}
\author{Juarez Sampaio}
\titlegraphic{\leavevmode\smash{\raisebox{5.5cm}{\includegraphics[width=0.2\textwidth]{logo}}}}
\institute{Rice University}
\date{\today}


\begin{document}
\begin{frame}
        \titlepage
\end{frame}

\begin{frame}[fragile]
\frametitle{During the last week:}
  \begin{itemize}
     \item machine learning approach to tracking
     \item environment to allow repeating the same experiment on different components of the image
     \item tracking problem solved by allowing user to correct tracking
  \end{itemize}
\end{frame}

\begin{frame}
  \frametitle{Machine Learning Approach to Tracking}
  \begin{block}{Motivation}
        Previous experiment with template matching tells us that simple similarity measures are not enough to track the
        mouse. On template matching we swipe the image comparing a block of image with one or more templates. What if
        instead of this, we had a classifier telling weather there is a mouse in the current block or not?
  \end{block}

  In order to do this:
  \begin{itemize}
      \item We need to collect positive and negative samples. Show program machineLearningExperiment/collectSampels.py.
      \item train a classifier
      \item apply it to tracking. Show demo machineLearningExperiment/experiment\_detextion.py.
    \end{itemize}
\end{frame}


\begin{frame}
  \frametitle{Machine Learning Approach to Tracking - Details}
  \begin{itemize}
      \item 60x60 images are manually cropped from the video
      \item every 60x60 sample is converted to a 30x30 version by low pass filtering followed by downsample
      \item this is know as a pyramid down operation
      \item the 30x30 image is made into a 1x900 vector
      \item the feature domain thus become 900 dimension
      \item A few classifiers were tried. The one the produced best result was Random Forest with 60 classifiers of deep
        10 each.
    \end{itemize}

\end{frame}

\begin{frame}
  \frametitle{Machine Learning Approach to Tracking - Results}
  \begin{itemize}
      \item Ensemble methods produced accuracy on test data pf 94\%
      \item even though, when applied to tracking, too many false positives
      breaks the tracking.
      \item the method is also really computational expensive and the tracking becomes slow
      \item classifying an image is actually really cheap. The expensive part is converting image samples to the desired
        format. For every 60x60 square in the domain we have to pyramid down it and convert it to 1x900 vector.
      \item multi processing approaches could improve performance
      \item Maybe more samples are needed since we're on a high dimensional space
      \item Or maybe, pixel intensity levels are not enough description. When our brain tracks the mouse it uses a lot
        of previous knowledge. 
    \end{itemize}
\end{frame}

\begin{frame}
  \frametitle{Environment to develop experiments on multiple components}
  \begin{itemize}
      \item I develop a interface that will help applying one algorithm do several components at the same time.
      \item show demo componentsExperiment/histogramEqualizationExperiment.py
    \end{itemize}
\end{frame}

\begin{frame}
  \frametitle{'Solving' the Tracking Problem}
  \begin{itemize}
      \item The conditions of the recordings do not allow us to rely on much. Geometry, color, spectrum, texture \ldots
        all will eventually change during the video. When such a change occurs, my trackers would get lost forever.
      \item My current solution is to ask the user to reset the tracking when this occurs
      \item Although not elegant, I cannot afford spend more time on tracking. It's time to start pay attention to
        whatever the mouse is doing
      \item I hope If we're able to show that we can detect seizures based only on video that future students can get
        better videos to allow easier tracking.
      \item show demo python/main.py
    \end{itemize}
\end{frame}
\end{document}
