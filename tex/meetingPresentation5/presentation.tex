\documentclass{beamer}
\usepackage[T1]{fontenc}
\usepackage[utf8]{inputenc}
\usepackage{lmodern}
\usepackage{graphicx}			%para imagens
\usepackage{epstopdf} 			%resolve problemas eps-pdf
\usepackage{fancyhdr}			% para o cabeçalho bonito
\usepackage{caption}				%para legendas
\usepackage{placeins} 			%controlar o lugar dos floats
\usepackage{color}
\usepackage{url}
\renewcommand{\UrlFont}{\tiny}
\usepackage{relsize}
\usepackage{hyperref}

\usetheme{Dresden}
\usecolortheme{orchid}
\setbeamertemplate{navigation symbols}{}

\newcommand{\HRule}{\rule{\linewidth}{0.5mm}}

\title{Summer Research Project \\  Video Based Mouse Seizure Detection}
\author{Juarez Sampaio}
\titlegraphic{\leavevmode\smash{\raisebox{5.5cm}{\includegraphics[width=0.2\textwidth]{logo}}}}
\institute{Rice University}
\date{\today}


\begin{document}
\begin{frame}
        \titlepage
\end{frame}

\begin{frame}[fragile]
  \frametitle{Content}
  \begin{itemize}
     \item A few concepts in image processing and computer vision:
       \begin{itemize}
           \item a few morphological operations
       \end{itemize}
     \item Results of the week
       \begin{itemize}
           \item Filtering Flow by connectivity
           \item Hand writing line detection criteria
       \end{itemize}
  \end{itemize}
\end{frame}


\section{Background Concepts}
\subsection{Morphological Transformation}

\begin{frame}
  \frametitle{\secname : \subsecname}
  \begin{definition}
    Morphological Transformations are simple transformations based on shape. They're usually performed on a binary input
    and require the definition of a \textbf{structuring element}, which decides the extent and nature of the operation.
  \end{definition}

  \begin{itemize}
      \item they're usually used to enhance some property of the image or to reduce noise.
      \item Morphological transformations are one of the techniques broadly used during this work.
      \item They can be seen as the \textbf{generalized convolution} where a window is slided through the image and some
        operation is realized at each pixel.
  \end{itemize}
\end{frame}

\begin{frame}
  \frametitle{\subsecname : Erosion}
  \begin{definition}
    The kernel is a matrix filled with ones and zeros. One means we care about the pixel position and zero means don't
    cares. Center the kernel at the pixel to be evaluated. If the all the positions marked as ones in the kernel are
    also ones in the image, the center pixel is one, otherwise it is zero.
  \end{definition}
  \begin{itemize}
      \item The kernel is usually a small matrix filled with ones. 
      \item It can also be a square matrix where the ones in it approximate an ellipse
      \item This operation will reduce the area of the foreground. If the foreground was initially noise and had small
        area, it will disappear.
  \end{itemize}
\end{frame}

\begin{frame}
      \begin{figure}
        \centering
        \includegraphics[width = 0.6\textwidth, keepaspectratio]{./eroisionSample.png}
        \caption{Effect of erosion with a 3x3 kernel. \footnote{from http://homepages.inf.ed.ac.uk/rbf/HIPR2/erode.htm}}
      \end{figure}
\end{frame}


\begin{frame}
  \frametitle{\subsecname : Dilation}
  \begin{definition}
    Similar to Erosion, but here the center pixel is set to one if at least one of the values in the image is one where
    the kernel is also one. 
  \end{definition}

  \begin{itemize}
    \item You can see erosion as an \textbf{and} between the kernel and the window and dilation as an \textbf{or}.(remember there
      are don't cares in the kernel)
    \item This will increase the area of what is considered foreground. Suppose there are holes in the foreground, it
      will likely close them with enough operations. 
    \item It will also enhance are of the noise signal. That's why we usually apply it followed by an erosion. This
      combination is called closing operation. The opposite order is called opening.
  \end{itemize}
\end{frame}


\begin{frame}
      \begin{figure}
        \centering
        \includegraphics[width = 0.6\textwidth, keepaspectratio]{./dilationSample.png}
        \caption{Effect of dilation with a 3x3 kernel \footnote{from http://homepages.inf.ed.ac.uk/rbf/HIPR2/dilate.htm}}
      \end{figure}
\end{frame}
\section{Results of the Week}
\subsection{Filtering Flow by Connectivity}

\begin{frame}
  \frametitle{\secname : \subsecname}
  \begin{block}{Motivation}
    Spurious optical flow differs from true movement flow in connectivity. Noisy flow is likely to occur on isolated
    points.
  \end{block}

  \begin{itemize}
      \item Connectivity is a term that remember us of graph theory and definitely some graph package could be used for
        this task
      \item However, there is an image processing trick that will do the job:
        \begin{itemize}
            \item Create a grid of zeros with the size of the image and fill with ones where the optical flow has magnitude in a
              certain range. 
            \item In case the flow is computed in sparse points, apply a closing operation to connect near points
            \item Look now for contours and keep only those with value of area in a certain range
        \end{itemize}
  \end{itemize}
\end{frame}


\begin{frame}
  \frametitle{\secname : \subsecname}
  \begin{itemize}
        \begin{itemize}
            \item Finally, keep only flow vectors on positions that lie inside one of the valid contours
            \item The only problem is that in the processes of closing we might increase the strength of a spurious
              signal. Setting the thresholds properly should solve this problem.
            \item \href{run:runFlowFilter.sh}{Run demo on camera}
        \end{itemize}
  \end{itemize}
\end{frame}

\subsection{Detecting Mouse in a Line}
\begin{frame}
  \frametitle{\secname: \subsecname}
  \begin{itemize}
      \item A few week ago I suggested that neural networks would be able to detect a mouse in a line of pixels by being
        able to distinguish between its texture and the background texture
      \item Still, the neural network wasn't able to learn the features; maybe for a bug in implementation, maybe for
        lack of layers and nodes
      \item But what if we're able to right down the features we wanted the neural net to learn?
  \end{itemize}
\end{frame}

\begin{frame}
  \frametitle{\subsecname : details}
  \begin{itemize}
      \item Fix a window of length L around one pixel
      \item If L is not too big, on a plain surface the values within the window should be similar
      \item On a shiny or noisy surface, the value inside the window will vary a lot
      \item The variance is a good measure of how much dispersion theres is in a set.
      \item \href{run:runLineDetect.sh}{Run demo of detection}
  \end{itemize}
\end{frame}

\end{document}


