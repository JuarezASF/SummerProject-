\documentclass{beamer}
\usepackage[T1]{fontenc}
\usepackage[utf8]{inputenc}
\usepackage{lmodern}
\usepackage{graphicx}			%para imagens
\usepackage{epstopdf} 			%resolve problemas eps-pdf
\usepackage{fancyhdr}			% para o cabeçalho bonito
\usepackage{caption}				%para legendas
\usepackage{placeins} 			%controlar o lugar dos floats
\usepackage{color}
\usepackage{url}
\renewcommand{\UrlFont}{\tiny}
\usepackage{relsize}
\usepackage{hyperref}

\usetheme{Dresden}
\usecolortheme{orchid}
\setbeamertemplate{navigation symbols}{}

\newcommand{\HRule}{\rule{\linewidth}{0.5mm}}

\title{Summer Research Project \\  Video Based Mouse Seizure Detection}
\author{Juarez Sampaio}
\titlegraphic{\leavevmode\smash{\raisebox{5.5cm}{\includegraphics[width=0.2\textwidth]{logo}}}}
\institute{Rice University}
\date{\today}


\begin{document}
\begin{frame}
        \titlepage
\end{frame}

\begin{frame}[fragile]
  \frametitle{Content}
  \begin{itemize}
     \item A few concepts in image processing and computer vision:
       \begin{itemize}
           \item Code Vectorization
       \end{itemize}
     \item Results of the week
       \begin{itemize}
           \item Collecting Labeled Sequences
           \item Filtering Flow
       \end{itemize}
  \end{itemize}
\end{frame}


\section{Background Concepts}
\subsection{Code Vectorization}

\begin{frame}
  \frametitle{\secname : \subsecname}
  \begin{block}{Motivation}
    When dealing with matrix data we might be tempted to write for loops after for loops. However, most operation on an
    element of the matrix are independent of the others. Those independent operations could take place in parallel.
    Writing parallel code is troublesome due to memory access conflicts. However, your code can run in parallel if you
    are able to use methods already implemented in parallel. 
  \end{block}

  \begin{itemize}
      \item Part of code vectorization is the art of written for loops as matrix and vectors products and dot products.
      \item Other techniques are useful: you might want to generate all index of a matrix that are going to be modified
        and then modify them all at once.
  \end{itemize}
\end{frame}

\begin{frame}[fragile]
  \frametitle{\secname : \subsecname}
  \begin{itemize}
      \item In order to learn code vectorization you need to get familiar with the tools provided by the library(in this
        case, numpy)
      \item example from my magnitude band pass flow filter
        \begin{scriptsize}
\begin{verbatim}
flowNorm = np.linalg.norm(newP - oldP, axis = 1)
valid = np.where(np.logical_and(flowNorm >= self.low_th, \
                        flowNorm <= self.upper_th))
\end{verbatim}
        \end{scriptsize}
      \item  \href{run:runVectorization.sh}{Run demo}
    \end{itemize}

\end{frame}
\section{Results of the Week}
\subsection{Collecting Labeled Sequences}

\begin{frame}[fragile]
  \frametitle{\secname : \subsecname}
  \begin{alertblock}{Motivation}
    On Supervised Machine Learning we need to train the classifier with labeled samples. The samples collected should be
    independently draw and be representative of the space of possible samples.
  \end{alertblock}

  \begin{itemize}
     \item The previous data available had resolution of minutes
     \item We plan to detect seizure within sequences of seconds(Is there a seizure in this 3(or 5) seconds video)
     \item We got together with an expert on this experiment and asked him to precisely point the seizures
  \end{itemize}
\end{frame}


\begin{frame}[fragile]
  \frametitle{\secname : \subsecname}

  \begin{itemize}
     \item Because we wanted this made as quickly as possible two programs were written:
       \begin{itemize}
           \item One to crop the video around the minute indicated in the currently available data
           \item One to open the cropped video and with sliders to help select the seizure sequence.
      \end{itemize}
      \item For the first program it was important that the videos were names with the initial time and also that the
        frame rate is constant along the video
      \item Example of data collected:
        \begin{itemize}
        \begin{scriptsize}
          \item \begin{verbatim} video 2014-07-16_08-55.avi seizure from frame 01075 to frame 01210 \end{verbatim}
        \end{scriptsize}
        \end{itemize}
      \item  \href{run:runCollect.sh}{Run demo}
  \end{itemize}
\end{frame}


\subsection{Filtering Optical Flow}
\begin{frame}[fragile]
  \frametitle{\secname : \subsecname}
  \begin{alertblock}{Motivation}
    Previously it was noted that spurious optical flow is often present on the data. However, true movement differs from
    spurious flow on magnitude and connectivity of the flow.
  \end{alertblock}

  \begin{itemize}
     \item A filter was implemented using the first idea: a low pass filter in magnitude will reduce the amount of
       spurious data.
     \item Because we wanted to study the performance of the filter being implemented, the problem was first reproduced
       on another environment.
     \item  \href{run:runFlowFilter.sh}{Run demo on camera}
     \item  \href{run:runFlow.sh}{Run demo on main program}
  \end{itemize}
\end{frame}
\end{document}
